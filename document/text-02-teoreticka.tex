\chapter{Teoretická část}


\section{Virtualizace}
Virtualizace~\autocite{virtualizace-Martin-Polednik} je technologie, která se využívá pro efektivnější rozdělení hardwarových zdrojů fyzického počítače mezi virtualizované/hostované počítače.
Jednotlivé systémy běží v podstatě na jiném hardwaru, díky čemuž jsou jejich procesy izolovány, což přispívá k bezpečnosti řešení.
Virtuální stroj je zárověň jednodužší spravovat, díky tomu, že máme úplnou kontrolu nad daným virtuálním strojem, což pomáhá například při klonování nebo restartu.


\section{Kontejnerizace}
Kontejnerizace~\autocite{virtualizace-Martin-Polednik} je technologie umožňující zapouzdření funkcionality jedné aplikace společně se všemi jejími závislostmi a zdroji do jednotky nazývané česky obrazy.
Konkrétní běžící instance jednoho obrazu se nazývá kontejner.
Tyto kontejnery pak lze spouštět v prakticky neomezeném množsví nezávisle na platformě a výhodou je, že jsou jejich běhy navzájem izolované, takže běh jednoho neovlivní ostatní.
Je to v podstatě forma virtualizace se všemi jejími výhodami, s tím rozdílem, že daný kontejner obsahuje jen nezbytné věci pro běh dané aplikace.

\subsection{Docker Engine}
Jedna z technologií používaných pro kontejnerizaci je Docker Engine~\autocite{kontejnerizace-docker}.
Používá se pro tvorbu, správu, orchestraci, verzování a nasazování jednotlivých kontejnerů.

\subsection{Docker Compose}
Docker compose~\autocite{kontejnerizace-docker-compose} je nástroj pro Docker, který nám umožňuje tvořit komplikované ekosystémy jednotlivých kontejnerů, jež spolu v rámci něho mohou spolupracovat.
Pomáhá nám například se síťováním nebo konfigurací kontejnerů.


\section{Mikroservisní architektura}\label{sec:microservice-architecture}
Mikroservisní architektura je přístup k vývoji softwarových aplikací, kdy je celek rozdělen na malé, nezávislé služby nazývané mikroservisy.
Každá služba běží samostatně, komunikuje s ostatními službami pomocí dobře definovaného API a je zaměřena na konkrétní funkcionalitu.
Tento model umožňuje snadnější údržbu, škálovatelnost a flexibilitu systému.


\section{Jazyk Python}\label{sec:python}
Python je vysokoúrovňový programovací jazyk známý svou jednoduchou a čitelnou syntaxí.
Podporuje několik programovacích paradigmat, včetně objektově orientovaného, procedurálního a funkcionálního programování.
Python je široce používán v různých oblastech, jako je webový vývoj, vědecké výpočty, umělá inteligence, strojové učení a datová analýza.
Disponuje rozsáhlou standardní knihovnou a množstvím externích modulů, které usnadňují vývoj komplexních aplikací.
Je multiplatformní, což znamená, že programy v Pythonu lze spouštět na různých operačních systémech.


\section{Jazyky Wire a C++ }

\subsection{Wiring}\label{sec:wiring}
Wiring je open-source programovací jazyk a vývojové prostředí určené pro práci s mikrokontroléry.
Jeho cílem je usnadnit programování elektronických zařízení a interaktivních projektů, zejména pro umělce, designéry a studenty.
Syntaxe Wiringu vychází z jazyka C++ a je navržena tak, aby byla snadno pochopitelná i pro začátečníky.
Wiring poskytuje intuitivní prostředí pro psaní kódu, jeho kompilaci a nahrávání přímo do mikrokontroléru.

\subsection{C++}\label{sec:cpp}
C++ je výkonný, objektově orientovaný programovací jazyk vyvinutý jako rozšíření jazyka C.
Umožňuje kombinovat nízkoúrovňové programování blízké hardwaru s vysokou úrovní abstrakce, což z něj činí univerzální nástroj pro různé typy softwaru.
C++ podporuje více programovacích paradigmat, včetně procedurálního, objektově orientovaného a generického programování.
Je široce využíván pro vývoj systémového softwaru, her, grafiky, aplikací s vysokým výkonem a vestavěných systémů.


\section{Git}\label{sec:git}
Git je systém pro správu verzí, který umožňuje sledovat změny v zdrojovém kódu během vývoje softwaru.
Tento systém se stal již dnes standardem moderního vývoje softwaru.
Vytvořil ho Linus Torvalds v roce 2005 pro potřeby vývoje jádra operačního systému Linux.
Git umožňuje více vývojářům pracovat současně na jednom projektu bez rizika přepsání či ztráty práce.
Každá kopie repozitáře obsahuje kompletní historii projektu, což zajišťuje možnost vrátit se k předchozím verzím a pracovat i bez připojení k internetu.
Důležité funkce Gitu zahrnují rychlost, efektivitu při práci s velkými projekty a podporu nelineárního vývoje prostřednictvím větvení.


\section{Github workflow}\label{sec:github-workflow}
GitHub Workflow je nástroj pro automatizaci procesů při vývoji softwaru na platformě GitHub.
Umožňuje vytvářet a spravovat tzv. workflow pomocí souborů ve formátu YAML, které definují jednotlivé kroky.
Tyto kroky nebo také akce, se mohou spouštět na základě různých událostí, jako je push kódu nebo vytvoření pull requestu.
GitHub Workflow podporuje kontinuální integraci a nasazení (CI/CD), automatické testování, nasazení aplikací a další úlohy.


\section{Mqtt}\label{sec:mqtt}
MQTT (Message Queuing Telemetry Transport) je lehký messagingový protokol pro publikování a odebírání zpráv, navržený pro komunikaci mezi zařízeními v sítích s omezenou kapacitou nebo vysokou latencí.
Je často využíván v oblasti Internetu věcí (IoT) pro přenos dat mezi senzory, akčními členy a centrálními systémy.
MQTT pracuje na principu architektury klient–server, kde klienti(publisheři) publikují zprávy na určité téma(topic) a broker tyto zprávy distribuuje odběratelům(subscriberům), kteří jsou na dané téma přihlášeni.


\section{Home Assistant}\label{sec:home-assistant}
Home Assistant je open-source platforma pro automatizaci domácnosti napsaná v jazyce Python.
Umožňuje centrálně ovládat a monitorovat různá zařízení v chytré domácnosti, jako jsou osvětlení, termostaty, bezpečnostní systémy a další IoT zařízení.
Home Assistant podporuje integraci s více než tisíci různými komponentami a službami, což umožňuje vytvořit komplexní a přizpůsobené automatizační scénáře.
Platforma běží lokálně, což zajišťuje vyšší úroveň soukromí a nezávislost na cloudových službách.
Uživatelé mohou využít webové rozhraní pro správu a nastavování automatizací nebo psát vlastní konfigurace pomocí YAML souborů.


\section{Arduino}\label{sec:arduino}
Arduino je open-source platforma pro prototypování elektroniky založená na snadno použitelném hardwaru a softwaru.
Skládá se z mikroprocesorové desky a vývojového prostředí Arduino IDE, které využívá jazyk podobný C/C++.
Arduino desky umožňují komunikaci s různými senzory a akčními členy, což usnadňuje tvorbu interaktivních projektů.
Podporuje řadu rozšiřujících modulů, tzv. shieldů, které rozšiřují jeho funkčnost například o bezdrátovou komunikaci, ovládání motorů či připojení k internetu.
Arduino usnadňuje rychlý vývoj a testování elektronických aplikací bez hlubokých znalostí elektroniky.


\section{Ip kamera a RTSP}\label{sec:ipcamera-rtsp}
IP kamera je digitální zařízení, které přenáší obraz a zvuk přes IP sítě, jako je internet nebo lokální síť.
To umožňuje vzdálený přístup k živému vysílání nebo záznamům bez nutnosti speciálního kabelového připojení.
Pro efektivní přenos multimediálních dat v reálném čase se často využívá protokol RTSP (Real Time Streaming Protocol).

\subsection{RTSP}\label{sec:rtsp}
RTSP je síťový protokol, který umožňuje kontrolu nad streamováním médií, jako je spouštění, zastavování nebo přetáčení videa.


\section{Cloudflare tunneling}\label{sec:cf-tunnel}
Cloudflare Tunneling je služba poskytovaná společností Cloudflare, která umožňuje bezpečné a jednoduché propojení lokálního serveru s internetem.
Využívá tunelování k vytvoření šifrovaného spojení mezi vaším interním systémem a infrastrukturou Cloudflare bez nutnosti otevírat porty na firewallu nebo nastavovat složité síťové konfigurace.
To usnadňuje publikování webových aplikací nebo služeb hostovaných na lokálních serverech, aniž by byla odhalena jejich skutečná IP adresa.


\section{Flask}\label{sec:flask}
Flask je lehký webový framework pro Python, který umožňuje rychlé a jednoduché vytváření webových aplikací.
Patří mezi mikroframeworky, což znamená, že poskytuje pouze základní funkce potřebné pro webový vývoj, jako je směrování URL a zpracování HTTP požadavků.
Díky své modularitě umožňuje vývojářům přidávat rozšíření a knihovny podle potřeby, například pro práci s databázemi, autentizaci či validaci.
Flask využívá šablonovací systém Jinja2 pro vytváření dynamických HTML stránek a nástroj Werkzeug pro WSGI kompatibilitu.


\section{APScheduler}\label{sec:apscheduler}
APScheduler (Advanced Python Scheduler) je knihovna pro programovací jazyk Python, která umožňuje plánovat a spouštět úlohy v určených časech nebo intervalech.
Poskytuje nástroje pro definování plánů úloh, jako jsou jednorázové spuštění, opakované intervaly nebo specifické časové výrazy.


\section{Paho-mqtt}\label{sec:paho-mqtt}
Paho-MQTT je oficiální klientská knihovna pro protokol MQTT určená pro programovací jazyk Python.
Tato knihovna umožňuje aplikacím v Pythonu komunikovat s MQTT brokerem, což je centrální bod pro výměnu zpráv v rámci MQTT messaging systému.


\section{Python-dotenv}\label{sec:python-dotenv}
Python-dotenv je knihovna pro programovací jazyk Python, která umožňuje načítat proměnné prostředí ze souboru.env.
Tento soubor obsahuje konfiguraci ve formátu klíč–hodnota, kde jsou uloženy citlivé informace, jako jsou hesla, API klíče nebo nastavení aplikace.


\section{Pyserial}\label{sec:pyserial}
PySerial je knihovna pro programovací jazyk Python, která umožňuje komunikaci přes sériové porty. Poskytuje jednotné rozhraní pro přístup k sériovým portům na různých operačních systémech, jako jsou Windows, Linux nebo macOS. PySerial umožňuje otevírání, čtení a zápis dat do sériového portu přímo z Python aplikací.


\section{Strojové učení}\label{sec:ml}
Strojové učení je oblast umělé inteligence zaměřená na tvorbu algoritmů a modelů, které umožňují počítačům se samostatně učit z dat bez explicitního naprogramování každého kroku. Využívá statistických metod a matematických modelů k analýze velkého množství dat s cílem odhalit vzory a vztahy. Tyto vzory pak slouží k predikci nebo rozhodování v různých situacích. Strojové učení se dělí na učení s učitelem, bez učitele a posilované učení. Praktické aplikace zahrnují rozpoznávání hlasu a obrazu, doporučovací systémy, detekci podvodů či autonomní řízení vozidel. Díky strojovému učení mohou systémy neustále zlepšovat svou výkonnost na základě nových dat.


\section{Yolo Ultralytics}\label{sec:yolo-ultralytics}
Ultralytics YOLO je pokročilý systém pro detekci objektů v reálném čase založený na hlubokém učení. Jedná se o implementaci architektury YOLO (You Only Look Once), kterou vyvinula společnost Ultralytics. Tento model využívá konvoluční neuronové sítě k analýze obrazových dat a současné identifikaci více objektů během jediného průchodu sítí. Díky své vysoké rychlosti a přesnosti je ideální pro aplikace náročné na čas, jako je autonomní řízení, bezpečnostní systémy nebo analýza videa. Ultralytics YOLO je dostupný jako open-source software, což umožňuje jeho široké využití ve výzkumu i v průmyslových aplikacích.


\section{Tenzometrický senzor}\label{sec:tenzo}
Tenzometrický senzor je zařízení sloužící k měření mechanického napětí nebo deformace v materiálu či konstrukci. Základním principem tenzometru je využití změny elektrického odporu vodivého materiálu při jeho mechanickém zatížení. Nejčastěji se používají tenzometry s kovovou fólií nebo drátkem, který je pevně spojen s měřeným objektem. Když je objekt zatížen silou, dojde k jeho deformaci, což způsobí prodloužení nebo zkrácení tenzometru a tím i změnu jeho elektrického odporu. Tato změna je úměrná velikosti aplikovaného napětí a může být přesně změřena pomocí elektrických obvodů, jako je Wheatstoneův můstek. Tenzometrické senzory nacházejí uplatnění v oblastech jako je strojírenství, stavebnictví, letectví či při vývoji nových materiálů, kde je důležité sledovat mechanické vlastnosti a bezpečnost konstrukcí.


\section{Power over Ethernet (PoE)}\label{sec:poe}
Power over Ethernet (PoE) je technologie umožňující přenos elektrické energie společně s daty prostřednictvím standardního ethernetového kabelu typu kroucená dvojlinka. Tato technologie umožňuje napájet síťová zařízení, jako jsou IP kamery, bezdrátové přístupové body nebo VoIP telefony, bez potřeby samostatného napájecího zdroje. PoE využívá nevyužité páry vodičů v kabelu nebo kombinuje napájení s datovými signály na stejných vodičích. Existují různé standardy PoE, jako IEEE 802.3af, 802.3at a 802.3bt, které definují maximální výkon a kompatibilitu zařízení. Použití PoE zjednodušuje instalaci, snižuje náklady na kabeláž a umožňuje flexibilnější umístění zařízení bez závislosti na elektrických zásuvkách.


\section{Wifi extender}\label{sec:wifi-extender}
Wi-Fi extender, také známý jako repeater nebo zesilovač signálu, je zařízení určené k rozšíření dosahu bezdrátové sítě. Přijímá existující Wi-Fi signál z routeru a znovu ho vysílá do oblastí s nedostatečným pokrytím. Tím eliminuje \"mrtvé zóny\" v domácnosti nebo kanceláři, kde je signál slabý nebo žádný. Instalace je obvykle jednoduchá a nevyžaduje dodatečné kabely. Wi-Fi extendery podporují různé standardy Wi-Fi, jako 802.11n nebo 802.11ac, a nabízejí přenosové rychlosti odpovídající těmto standardům. Pro efektivní rozšíření sítě je důležité umístit extender tam, kde ještě přijímá silný signál z routeru.


\section{TailScale vpn}\label{sec:tailscale}
Tailscale VPN je moderní služba pro vytváření virtuálních privátních sítí, která využívá protokol WireGuard k zajištění bezpečné komunikace mezi zařízeními. Umožňuje snadné nastavení sítě bez složitých konfiguračních procesů tradičních VPN řešení. Tailscale vytváří šifrované peer-to-peer spojení mezi zařízeními na základě jejich identity, spravované prostřednictvím cloudu. To umožňuje uživatelům bezpečně přistupovat k interním sítím a službám odkudkoli na světě. Díky automatické správě síťových konfigurací a firewallu snižuje nároky na údržbu a zvyšuje celkovou bezpečnost. Tailscale je vhodný pro jednotlivce, týmy i organizace hledající efektivní a jednoduché VPN řešení pro propojení svých zařízení.


\section{Raspberry PI}\label{sec:rpi}
Raspberry Pi je řada malých jednočipových počítačů vyvinutých nadací Raspberry Pi ve Velké Británii s cílem podpořit výuku informatiky a programování. Tyto cenově dostupné zařízení nabízejí plnohodnotné funkce počítače na kompaktním hardwaru. Raspberry Pi je vybaven procesorem ARM, grafickým výstupem HDMI, USB porty, Ethernetem a dalšími rozhraními pro připojení periferií a senzorů. Nejčastěji na něm běží operační systém Raspbian, který je založen na Linuxu. Díky své univerzálnosti a přístupnosti je široce využíván nejen ve vzdělávání, ale i v projektech IoT, domácí automatizace, robotiky a dalších oblastech vyžadujících flexibilní a výkonnou výpočetní platformu.


\section{Raspberry PI OS}\label{sec:rpi-os}
Raspberry Pi OS je oficiální operační systém pro počítače Raspberry Pi, vyvinutý nadací Raspberry Pi Foundation. Je založen na distribuci Debian Linux a je optimalizován pro hardware Raspberry Pi. Systém poskytuje stabilní a uživatelsky přívětivé prostředí s podporou široké škály aplikací. Raspberry Pi OS obsahuje předinstalované nástroje pro výuku programování, jako jsou Python, Scratch či Sonic Pi, což podporuje vzdělávací poslání platformy. Nabízí grafické uživatelské rozhraní, ale i možnost běhu v terminálu pro pokročilé uživatele. Díky pravidelným aktualizacím a široké komunitní podpoře je ideální volbou pro projekty v oblasti IoT, domácí automatizace či robotiky.


\section{Yaml}\label{sec:yaml}
YAML (YAML Ain't Markup Language) je formát pro serializaci dat navržený pro snadnou čitelnost a zápis člověkem. Používá se převážně pro zápis konfigurací v různých aplikacích a systémech. Jeho syntaxe je založena na odsazení a struktuře klíč-hodnota, což umožňuje vytvářet přehledná a hierarchicky uspořádaná data. YAML podporuje různé datové typy, jako jsou řetězce, číselné hodnoty, seznamy a slovníky. Díky své jednoduchosti a flexibilitě je oblíbený mezi vývojáři a administrátory. Využívá se například v nástrojích pro správu kontejnerů, jako je Docker Compose, nebo v automatizačních systémech typu Ansible pro definování infrastruktury jako kódu.


\section{GUI}\label{sec:gui}
GUI (Grafické uživatelské rozhraní) je způsob interakce člověka s počítačem pomocí grafických prvků, jako jsou ikony, tlačítka, menu a okna. Na rozdíl od textového rozhraní, kde uživatel zadává příkazy prostřednictvím textu, GUI umožňuje ovládání systému pomocí myši, dotykové obrazovky či jiného ukazovacího zařízení. Tento přístup zvyšuje intuitivnost a uživatelskou přívětivost aplikací a operačních systémů. GUI je založeno na konceptu WIMP (Window, Icon, Menu, Pointing device), který standardizuje prvky rozhraní pro konzistentní uživatelský zážitek. Vývoj grafických rozhraní vyžaduje znalost programovacích jazyků a knihoven, jako jsou Qt, GTK nebo Windows API.


\section{HTTP a REST}\label{sec:http-rest}
HTTP (Hypertext Transfer Protocol) je základní komunikační protokol pro web, který umožňuje přenos dat mezi klientem a serverem pomocí žádostí a odpovědí.

\subsection{REST (Representational State Transfer)}
REST(Representational State Transfer) je architektonický styl pro tvorbu webových služeb, který využívá protokol HTTP. REST definuje sadu principů pro uspořádání API tak, aby byly škálovatelné, flexibilní a snadno použitelné. V rámci RESTful služeb se využívají standardní HTTP metody jako GET, POST, PUT a DELETE k manipulaci s prostředky. Tento přístup usnadňuje komunikaci mezi různými systémy a umožňuje efektivní integraci aplikací v rámci webu.


\section{Latex}\label{sec:latex}
LaTeX je vysoce kvalitní systém pro sazbu dokumentů, který umožňuje tvorbu profesionálně vypadajících vědeckých a technických textů. Je založen na systému TeX, který vyvinul Donald Knuth. LaTeX využívá značkovací jazyk, ve kterém uživatelé definují strukturu dokumentu pomocí příkazů a prostředí. Tento přístup umožňuje automatickou správu číslování kapitol, odkazů, bibliografií a tvorbu obsahu. LaTeX je oblíbený zejména v akademickém prostředí pro svou schopnost precizně sázet matematické vzorce a rovnice. Díky své flexibilitě je vhodný pro vytváření diplomových prací, odborných článků, knih a dalších publikací, kde je kladen důraz na typografickou kvalitu.


\section{Backend}\label{sec:backend}
Backend je část softwarové aplikace, která běží na serveru a stará se o logiku, zpracování dat a komunikaci s databázemi. Je to neviditelná vrstva pro uživatele, ale klíčová pro funkčnost aplikace. Backend přijímá požadavky od frontendu (uživatelského rozhraní), zpracovává je a odesílá zpět potřebné informace.


\section{Environment proměnné}\label{sec:environment-variables}
Environment proměnné jsou nastavení v operačním systému, která ovlivňují chování běžících programů a skriptů. Umožňují předávat důležité informace, jako jsou cesty k souborům, konfigurace nebo přístupové údaje, bez potřeby je pevně zakomponovat do kódu.




